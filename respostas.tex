\documentclass{article}
\usepackage{amssymb}
\usepackage{graphicx} 
\usepackage{amsmath}
\usepackage{amsfonts}
\usepackage[a4paper, left=2cm, right=2cm, top=2cm, bottom=2cm]{geometry}

\title{Lista 3 Fundamentos Matemáticos da Computação II - Respostas}


\begin{document}

\maketitle

\begin{center}
Alesandro Alex Mendes da Silva \\
Francisco Matheus Fonseca de Farias \\
Sávio Emanuel Mariano Fonseca \\
Sebastião Fellipe Pinto Lopes \\
Weuler dos Santos Barbosa \\
\end{center}


\section{Seção Múltipla Escolha}

\subsection{Questão M1} 
Considerando a definição de um grupo abeliano e a operação
*: $\mathbb{R}\times\mathbb{R} \xrightarrow{} \mathbb{R}$, $(a,b) \mapsto a+b-3$, o grupo ($\mathbb{R},*)$ é abeliano, $e=3$ e o inverso do elemento 15 é o -9. \\

Dessa forma, a alternativa correta é a letra \textbf{B}


\subsection{Questão M2}
Dada a estrutura $(\mathbb{Z}, \ast)$, onde $\ast$ é definida por $a \ast b = a - b$, analisamos as propriedades:\\
\textbf{Fechamento:}  
A operação é fechada, pois $a - b \in \mathbb{Z}$ para quaisquer $a, b \in \mathbb{Z}$.\\
\textbf{Associatividade:}  
Não é associativa, pois $(a \ast b) \ast c = a - b - c \neq a - b + c = a \ast (b \ast c)$.\\
\textbf{Elemento neutro:}  
O elemento neutro é $0$, pois $a \ast 0 = a$.\\
\textbf{Inversos:}  
Não existem inversos distintos, pois $a \ast b = 0 \implies b = a$.\\

Portanto, a alternativa correta é \textbf{B}.

\subsection{Questão M3}
\[
\begin{array}{c|c|c|c|c|c|c}
+_6 & 0 & 1 & 2 & 3 & 4 & 5 \\
\hline
0 & 0 & 1 & 2 & 3 & 4 & 5 \\
1 & 1 & 2 & 3 & 4 & 5 & 0 \\
2 & 2 & 3 & 4 & 5 & 0 & 1 \\
3 & 3 & 4 & 5 & 0 & 1 & 2 \\
4 & 4 & 5 & 0 & 1 & 2 & 3 \\
5 & 5 & 0 & 1 & 2 & 3 & 4 \\
\end{array}
\]

No grupo \( (\mathbb{Z}_6, +_6) \), o inverso do elemento \( \bar{4} \) é \( \bar{2} \), pois:
\[
4 + 2 \equiv 0 \pmod{6}.
\]

Logo, a alternativa correta é a letra \textbf{C}

\subsection{Questão M4}
Analisando a tábua de operação de * e o conjunto $A = \{1, 2, 3, 4, 5\}$, temos que:
\[
(3*3)*(4*4) = 1 = (4*4)*(3*3) 
\]

Logo, a operação é comutativa e alternativa correta é a letra \textbf{D}.

\vspace{0.5em}
\hrule
\vspace{0.5em}

\section{Seção Discursiva}

\subsection{Questão D1}
Considere a estrutura algébrica $(M_2(\mathbb{Z}), \cdot)$, onde $M_2(\mathbb{Z})$ é o conjunto das matrizes $2 \times 2$ com entradas inteiras e $(\cdot)$ é o produto usual de matrizes.

\begin{enumerate}
    \item[\textbf{A.}] \textbf{A operação do produto de matrizes satisfaz a condição de fechamento? Explique.}

    Sim, a operação de produto de matrizes em $M_2(\mathbb{Z})$ satisfaz a condição de fechamento. Dado que o produto usual de duas matrizes $2 \times 2$ com entradas inteiras resulta em outra matriz $2 \times 2$ com entradas inteiras, concluímos que o resultado pertence ao conjunto $M_2(\mathbb{Z})$.

    \item[\textbf{B.}] \textbf{A operação do produto de matrizes é associativa? Explique.}

    Sim, a operação de produto de matrizes é associativa. Para quaisquer matrizes $A, B, C \in M_2(\mathbb{Z})$, temos:
    \[
    A \cdot (B \cdot C) = (A \cdot B) \cdot C.
    \]
    Essa propriedade é válida para o produto usual de matrizes.

    \item[\textbf{C.}] \textbf{Verifique a existência de elemento neutro. Em caso afirmativo, especifique o elemento neutro.}

    Sim, existe um elemento neutro na operação de produto de matrizes. Esse elemento é a matriz identidade $I$, dada por:
    \[
    I = \begin{bmatrix}
    1 & 0 \\
    0 & 1
    \end{bmatrix}.
    \]
    Para qualquer matriz $A \in M_2(\mathbb{Z})$, temos:
    \[
    A \cdot I = I \cdot A = A.
    \]

    \item[\textbf{D.}] \textbf{Verifique a condição de existência de simétricos/inversos.}

    Nem todas as matrizes $2 \times 2$ com entradas inteiras possuem inverso em $M_2(\mathbb{Z})$. Uma matriz $A \in M_2(\mathbb{Z})$ é inversível se, e somente se, seu determinante $\det(A)$ for igual a $\pm 1$. Caso contrário, não existe matriz inversa em $M_2(\mathbb{Z})$.

    \item[\textbf{E.}] \textbf{Essa estrutura se enquadra na definição de grupos? Explique.}

    Não, a estrutura $(M_2(\mathbb{Z}), \cdot)$ não se enquadra na definição de grupo. Apesar de satisfazer as propriedades de fechamento, associatividade e possuir um elemento neutro, nem todos os elementos de $M_2(\mathbb{Z})$ possuem inverso, o que impede que a estrutura seja um grupo.

\end{enumerate}
\subsection{Questão D2}
Considere a estrutura algébrica $(\mathbb{R}^+, \ast)$ com a operação binária definida como $a \ast b = |a - b|$

\begin{enumerate}
    \item[\textbf{A.}] \textbf{Verifique se essa operação satisfaz a condição de fechamento.Explique.}
    Para a operação \( a \ast b = |a - b| \), observe que:
    \[
    a \in \mathbb{R}^+, \, b \in \mathbb{R}^+ \implies a - b \in \mathbb{R} \quad \text{e} \quad |a - b| \in \mathbb{R}^+ \cup \{0\}.
    \]
    Portanto, a operação \( \ast \) satisfaz a condição de fechamento sobre \( \mathbb{R}^+ \), pois o resultado \( |a - b| \) será um número real não negativo.
    
    \item[\textbf{B.}] \textbf{A operação possui elemento neutro? Explique com base na definição de elemento neutro e em caso positivo, descreva quem é o elemento neutro.}

    Para verificar a existência de um elemento neutro \( e \in \mathbb{R}^+ \), precisamos que:
    \[
    a \ast e = |a - e| = a, \quad \forall a \in \mathbb{R}^+.
    \]
    Temos que:
    \[
    a - e = a \implies e = 0.
    \]

    Então como \( e \in \mathbb{R}^+ \), portanto, existe um elemento neutro para esta operação.
    \item[\textbf{C.}] \textbf{Verifique a existência de simétricos/inversos.}
    Para que exista um simétrico \( b \in \mathbb{R}^+ \) de \( a \), precisamos que:
    \[
    a \ast b = |a - b| = e, \quad \text{onde } e = 0.
    \]
    Resolvendo \( |a - b| = 0 \):
    \[
    a - b = 0 \implies b = a.
    \]
    Portanto, o único simétrico de \( a \) é o próprio \( a \).
    \item[\textbf{D.}] \textbf{Essa operação é associativa e comutativa? Explique usando a definição de associatividade e comutatividade.}
    
    \textbf{Comutatividade:} Verificamos se:
    \[
    a \ast b = b \ast a.
    \]
    Sabemos que:
    \[
    a \ast b = |a - b| \quad \text{e} \quad b \ast a = |b - a| = |a - b|.
    \]
    Portanto, \( a \ast b = b \ast a \), e a operação é comutativa.

    \textbf{Associatividade:} Verificamos se:
    \[
    (a \ast b) \ast c = a \ast (b \ast c).
    \]
    Calculando cada lado:
    \[
    a \ast b = |a - b|, \quad (a \ast b) \ast c = ||a - b| - c|,
    \]
    \[
    b \ast c = |b - c|, \quad a \ast (b \ast c) = |a - |b - c||.
    \]
    Por trivialidade, \( ||a - b| - c| \neq |a - |b - c|| \). Logo, a operação \( \ast \) não é associativa.
    \item[\textbf{E.}] \textbf{Essa estrutura pode ser considerada um grupo? Explique.}
    Para que a estrutura \( (\mathbb{R}^+, \ast) \) seja um grupo, as seguintes condições devem ser satisfeitas:
    \begin{itemize}
        \item Fechamento: Satisfeito, pois \( |a - b| \in \mathbb{R}^+ \).
        \item Elemento neutro: Existe, e é \( e = 0 \).
        \item Simétricos: Cada elemento \( a \in \mathbb{R}^+ \) é o seu próprio inverso.
        \item Associatividade: Não satisfeito, pois \( \ast \) não é associativa.
    \end{itemize}
    Portanto, a estrutura \( (\mathbb{R}^+, \ast) \) não pode ser considerada um grupo devido a falta de Associatividade.
\end{enumerate}

\end{document}
