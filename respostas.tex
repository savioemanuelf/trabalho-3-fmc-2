\documentclass{article}
\usepackage{amssymb}
\usepackage{graphicx} 
\usepackage{amsmath}
\usepackage{amsfonts}

\title{Lista 3 Fundamentos Matemáticos da Computação II - Respostas}


\begin{document}

\maketitle

\begin{center}
Alesandro Alex Mendes da Silva \\
Francisco Matheus Fonseca de Farias \\
Sávio Emanuel Mariano Fonseca \\
Sebastião Fellipe Pinto Lopes \\
Weuller dos Santos barbosa \\
\end{center}


\section{Seção Múltipla Escolha}

\subsection{Questão M1} 
Considerando a definição de um grupo abeliano e a operação
*: $\mathbb{R}\times\mathbb{R} \xrightarrow{} \mathbb{R}$, $(a,b) \mapsto a+b-3$, o grupo ($\mathbb{R},*)$ é abeliano, $e=3$ e o inverso do elemento 15 é o -9. \\

Dessa forma, a alternativa correta é a letra \textbf{B}


\subsection{Questão M2}
% SUA RESPOSTA DA M2
\subsection{Questão M3}
% SUA RESPOSTA DA M3
\subsection{Questão M4}
Analisando a tábua de operação de * e o conjunto $A = \{1, 2, 3, 4, 5\}$, temos que:
\[
(3*3)*(4*4) = 1 = (4*4)*(3*3) 
\]

Logo, a operação é comutativa e alternativa correta é a letra \textbf{D}.

\vspace{0.5em}
\hrule
\vspace{0.5em}

\section{Seção Discursiva}

\subsection{Questão D1}
% SUA RESPOSTA DA D1
\subsection{Questão D2}
% SUA RESPOSTA DA D2


\end{document}
